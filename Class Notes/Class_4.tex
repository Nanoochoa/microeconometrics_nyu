\documentclass[12pt]{article}
\usepackage{geometry}
\geometry{a4paper, right=25mm, left=25mm, top=20mm, bottom=20mm}
\usepackage{amsmath}
\usepackage{graphicx}
\usepackage{amsfonts}
\usepackage{amssymb}
\usepackage{float}
\usepackage[svgnames]{xcolor}
\usepackage{graphicx}
\usepackage{setspace}
\usepackage{theorem}
\usepackage{pdflscape}
\usepackage{mathtools}
\usepackage{setspace}
\usepackage[utf8]{inputenc}
\usepackage[english]{babel}
\usepackage{indentfirst}
\usepackage[none]{hyphenat}
\usepackage{amssymb}
\usepackage{multirow}
\usepackage{natbib}
\usepackage[final]{pdfpages}

\usepackage{xcolor}
\usepackage{hyperref}
\hypersetup{
	 colorlinks   = true,
     citecolor    = purple,
	 urlcolor	  = purple,
	 linkcolor	  = purple
	 }
\usepackage{subfig}

\setlength{\parindent}{2.5em}
\setlength{\parskip}{0.8em}
\renewcommand{\baselinestretch}{1.7}

\DeclareMathOperator{\E}{\mathbb{E}}

\usepackage{accents}
\newcommand{\ubar}[1]{\underaccent{\bar}{#1}}

\newtheorem{theorem}{Theorem}
\newtheorem{lemma}{Lemma}
\newtheorem{proposition}{Proposition}
\newtheorem{corollary}{Corollary}
\newtheorem{assumption}{Assumption}

\newenvironment{proof}[1][Proof]{\begin{trivlist}
\item[\hskip \labelsep {\bfseries #1}]}{\end{trivlist}}
\newenvironment{definition}[1][Definition]{\begin{trivlist}
\item[\hskip \labelsep {\bfseries #1}]}{\end{trivlist}}
\newenvironment{example}[1][Example]{\begin{trivlist}
\item[\hskip \labelsep {\bfseries #1}]}{\end{trivlist}}
\newenvironment{remark}[1][Remark]{\begin{trivlist}
\item[\hskip \labelsep {\bfseries #1}]}{\end{trivlist}}

\newcommand{\qed}{\nobreak \ifvmode \relax \else
      \ifdim\lastskip<1.5em \hskip-\lastskip
      \hskip1.5em plus0em minus0.5em \fi \nobreak
      \vrule height0.75em width0.5em depth0.25em\fi}

\onehalfspacing
%%%%%%%%%%%%%%%%%%%%%%%%%%%%%%%%%%%%%%%%%%%%%%%%%

\title{\Large{\textbf{Metrics Class notes}} \\
Class 4 and 5}

\begin{document}

\maketitle

The model 
\begin{equation*}
	y_i = e(z_i, \epsilon_i, \theta)
\end{equation*}
where $\epsilon \sim f(\cdot | z, \theta)$ which is the unobserved heterogeneity *often multidimensional). We have many inconditional moments



\textbf{Simulation-Based Methods}

\begin{equation*}
	\mathbb{E}[y^h | x, \theta] = \int e^h(x, \epsilon, \theta) f(\epsilon | x, \theta) d \epsilon \equiv m_h(z,\theta)
\end{equation*}

The conditional moment $E[y^h-m_h(x,\theta_0) | x] = 0$ is intractable because of multidimensionality of $\epsilon$ ($\geq 4$) or $e(\cdot)$ is intractable. But the condition imply many unconditional moment conditions:

\begin{equation*}
	\mathbb{E}[\psi_h(z) [y^h - m_h(z;\theta_0)]] = 0
\end{equation*}

If the problem is the dimensionality of $\epsilon$ then we can do the following
\begin{equation*}
	E[y^h| x, \theta] 0 \frac{1}{S} \sum^S_{s = 1} e^h(x_i, \epsilon_{is}; \theta) \frac{f()}{g()}  
\end{equation*}

.
.
.

\textbf{Back to Pakes and Pollard}

\begin{equation*}
	\mathbb{E}[\psi_h(z) [y^h - m_h(z;\theta_0)]] = 0
\end{equation*}

Stack this $H$ moments in one vector $G(\theta)$

\begin{equation*}
	G(\theta) = \int h(y,x,\theta) dP(y,x)
\end{equation*}

since we can not compute $m^h$, $h()$ is intractable. They assume that there exist a function $H$ such that

\begin{equation*}
	h(y,x;\theta) = \int H(y,x,\epsilon)	 dP(\epsilon | y, x)
\end{equation*}

If the problem is the dimensionality of $\epsilon$ 
\begin{equation*}
	H(y,x,\theta) = \psi_h(x) \left( y^h - e^h(x,\epsilon, \theta) \frac{f(\epsilon | x, \theta)}{g(\epsilon | x, \theta)}\right)
\end{equation*}
 
 where we used the fact that $\epsilon$ is independent of $y$ so $E[\epsilon | y,x,\theta] = E[\epsilon | x,\theta]$. So we can approximate
 
 \begin{equation*}
 	\hat{h}(y_i, x_i, \theta) = \frac{1}{S} \sum^S_{s = 1} H(y_i, x_i, \epsilon) 
 \end{equation*}

where $\epsilon_{is} \sim P(\cdot| y_i,x_i)$ iid. In the example, 

\begin{equation}
	\hat{h}(y_i, x_i, \theta) = \psi_h(x_i) \left[ y^h_i - \frac{1}{S} \sum^S_{s = 1} e^h(x_i, \epsilon_{is}, \theta) \frac{f(\epsilon_{is} | x_i, \theta)}{g(\epsilon_{is} | x_i, \theta)} \right]
\end{equation}

\textbf{SMM}

In GMM the idea is to $|| \frac{1}{n} \sum^N h(y_i,x_i; \theta || \simeq 0$. In SMM the idea es to replace $h$ by $\hat{h}$. 

\section*{Multinomial Choice Model (McFadden 89')}

Individual has $m \geq 2$ alternatives. Vector of covariates $[z_1, ..., z_m]$ and a random vector of individual weights $\alpha_i$. Utility from j-th alternative is $z_j'\alpha$ and chooses it if $z_j' \alpha_i \geq z_k' \alpha_i$ for all $k \neq j$. 

Individual unobserved heterogeneity is $\alpha_i \sim h(\eta, \theta_0)$ with $\eta \sim g(\cdot)$, where $h$ and $g$ are known. 

Multinomial probit is the case where $\alpha_i \sim N_h (\mu, \Sigma) = h(\eta,\mu,\Sigma)$. 

[I got lost here..] he explains why $\alpha$ is generates as a $k \times 1$ vector function  $h(\eta,\theta_0)$ of an r-dimensional random vector $\eta$ with known distribution. [Some discussion about independence that I did not understand ]

so... the utility from individual is

\begin{equation*}
	u_{ji} = z_j' \alpha_i = z_{1j} \alpha_{1i} + ... + z_h \alpha_{hi}
\end{equation*}

[He talks about limitations of multinomial logit]

if covariates are stacked in a $m \times k$ matrix $Z$, the choice is specified by the response vector
\begin{equation*}
	d = D[Z h(\eta, \theta_0)]
\end{equation*}

where D maps into $\{0,1\}^m$. Note that $d = y$ and $D = e(x,\epsilon,\theta)$. So let $\pi$

\begin{equation*}
	E[d|x, \theta] = \pi(z, \theta)
\end{equation*} 

Why is $\pi$ intractable? 

\begin{equation*}
	\pi(z, \theta) = \int_{dimK} D[Zh(\eta, \theta_0)] dP(\eta)
\end{equation*} 

so the problem is the dimensionality. 

\section*{Class 5}

Now we go back to example 4.2 of the paper. 

Individual utility from $j$ to agent $i$ gives
\begin{equation*}
	U_{i,j} = z_j' \alpha_i
\end{equation*}

The Random Utility Model establishes that $i$ choose alternative $j$ iff
\begin{equation*}
	U_{i,j} \geq U_{i,j'}, \forall j' \neq j
\end{equation*}

The unobserved heterogeneity is on $\alpha$
\begin{equation*}
	\alpha_i = h(\eta_i , \theta_0) \text{ where } \eta_i \sim g(\cdot)
\end{equation*}
with $h(\cdot)$ and $g(\cdot)$ are known. When $\alpha_i \sim \mathcal{N}(\mu_0, \Sigma_0)$ then we are in the Multinomial Probit model. 

Stacking everything in matrices 

\begin{equation*}
	\left[ \begin{matrix}
		u_{1i} \\
		...\\
		u_{ni}
	\end{matrix} \right] = Z \beta_i = Zh(\eta_i,\theta_0)
\end{equation*}

and 

\begin{equation*}
	d = \left[ \begin{matrix}
		d_{1i} \\
		...\\
		d_{ni}
	\end{matrix} \right] = D[Zh(\eta_i, \theta_0)]
\end{equation*}

Last class we said that this is intractable because of the dimensionality. Now we are going to see how we simulate here. We want to look at 

\begin{equation*}
	\pi(z,\theta) = E[d | z,\theta] = \int_{dim K+M} D[Z h(\eta, \theta_0)] g(\eta) d\eta 
\end{equation*}

where the dimensionality problem comes from the integral being of dimension $K+M$ (which also kills the option to do MLE). We want to look for the $\theta$ that does

\begin{equation*}
	E[d|Z,\theta_0] - \pi(Z,\theta_0)] = 0 
\end{equation*}
 
 We know that this condition implies the unconditional moment
 \begin{equation*}
	E[ W(Z) [ d - \pi(Z,\theta_0)]] = 0 
\end{equation*}
but now d and Z are random, so we integrate over the measure $dP(d,Z)$
 \begin{equation*}
	\underbrace{G(\theta)}_{k \times 1} = \int W(Z,\theta) [ d - \pi(Z,\theta)] dP  
\end{equation*}
where $G(\theta_0) = 0$. So now we are going to simulate $\pi$. We need to replace $ \pi(Z_i,\theta)$ by a simulation estimator. For each individual generate $s$ new random variables $\eta_{i1},...,\eta_{is}$ and replace 

\begin{equation*}
	 \pi(Z_i,\theta) \simeq \hat{\pi}_s(Z_i, \theta) = \frac{1}{s} \sum^s D[Z_i h(\eta, \theta)]
\end{equation*}
and we do 
\begin{equation*}
	G_n(\theta) = n^{-1} \sum^n_{i = 1} W(Z_i,\theta) [d_i - \hat{\pi}_s(Z_i, \theta)]  
\end{equation*}

this works for $s = 1$ but the precision of the estimator increase with $s$. 

[I was lost in the variance discussion]

\section*{Indirect inference - Gourieroux et al.}

Consider the dynamic model
\begin{eqnarray*}
	y_t = e(y_{t-1}, x_t, u_t, \theta) \\
	u_t = \phi(u_{t-1}, \epsilon_{t}, \theta), \theta\in \Theta \in \mathbb{R}^{\mathbb{P}}
\end{eqnarray*}














\end{document}
